\documentclass[a4Paper]{article}
\raggedbottom
\usepackage[margin=1in,footskip=.25in]{geometry}
\usepackage{graphicx}
\usepackage{xspace}
\usepackage{lipsum}


\usepackage{caption}
\usepackage{glossaries}

\usepackage{hyperref}
\urlstyle{same}

\usepackage{enumitem}
\setlist{leftmargin=*}
\setlength{\parskip}{12pt}
\begin{document}

\begin{titlepage}
%\setlength{\voffset}{-0.1in}
%\setlength{\headsep}{5pt}
%\setlength{\textheight}{650pt}
%titlepage
%\thispagestyle{empty}
\clearpage
\vspace*{\fill}
\begin{center}
%\begin{minipage}{0.75\linewidth}
    \centering
     
%=====================================================%
%---------------TITLE--------------------------%
%=====================================================%    
      
  {{\Large \textbf{Topic Analysis and Synthesis Report}\par}}
    \vspace{1cm}% 
     
%=====================================================%
%---------------Partial Fulfillment---------------------%
%=====================================================%
    \vspace{0.4cm}
    {
    \textbf{\large Software Project Management (SOEN 6481)} \par}
    \vspace{0.2cm}

%=====================================================%
%---------------Degree---------------------%
%=====================================================%

\vspace{0.5cm}
    {
    \textbf{\large Topic 123: Don't Elevate the Means Beyond the End} \par}
    \vspace{2mm}


    

%=====================================================%
%---------------AUTHOR'S NAME-------------------------%
%=====================================================%
\vspace{5mm}
    {\large by\par}
    \vspace{0.05cm}
    {\small {\textbf{Supradeep Danturti}}  (40226103)\par}

    \vspace{0.9cm}
    
%=====================================================%
%---------------Supervisor Name---------------------%
%=====================================================%    
%    {\Large \textbf{Doctor of Philosophy} \par}
%    \vspace{0.5cm}
    { Under the Supervision of \par}
    {\small \textbf{\textbf{Professor:{ Pankaj Kamthan}}}\par}

     

%=====================================================%
%---------------UNIVERSITY lOGO-----------------------%
%=====================================================%
\includegraphics[width=0.45\linewidth]{METRICSTICS/media/Concordia-university-logo.jpg}
%    \rule{0.4\linewidth}{0.15\linewidth}\par

    
%=====================================================%
%---------------DEPARTMENT'S NAME-------------------------%
%=====================================================%

    {\large \textbf{Department of Computer Science and Software Engineering}\par}
    \vspace{0.4cm}
%=====================================================%
%---------------DATE----------------------------------%
%=====================================================%
    
%    {\large {October 30, 2023}}
%\end{minipage}
\end{center}
\vfill % equivalent to \vspace{\fill}
\clearpage
\end{titlepage}

\tableofcontents 

\pagebreak
\section{Abstract}
The report delves into Seth Dobbs's thought-provoking article "Don't Elevate the Means Beyond the End," which scrutinizes the prevalent tendency within the technology industry to prioritize means, namely new technologies and processes, over the actual ends of solving business problems and achieving strategic goals. Dobbs posits that this inclination is largely fueled by a desire to continuously acquire new technological knowledge and a sense of detachment felt by development teams from the overarching success of their respective companies.

The analysis centers on two primary factors: the allure of embracing new technologies without discerning their direct alignment with business needs and the perceived disconnection between the development teams and the overall corporate success. Dobbs highlights how the eagerness to adopt new technologies sometimes overshadows the ultimate objective of addressing business challenges. This rush to implement the latest trends, such as microservices, serverless, or blockchain, often leads to a myopic focus on the means—implementing these technologies—instead of the intended end goal—solving business problems.

Moreover, Dobbs underscores the critical impact of the disconnection felt by development teams from their companies' broader goals. The resulting emphasis on what they can control, such as implementing new technologies, might steer them away from aligning their work with the company's strategic objectives.

The report provides comprehensive examples to substantiate Dobbs's arguments, illustrating instances where companies haphazardly adopted new technologies, leading to complex systems that were challenging to maintain and scale. It also highlights cases where misaligned Agile methodologies resulted in excessively bureaucratic and inefficient processes.

In conclusion, the report echoes Dobbs's assertion that genuine success emerges from a deliberate understanding of when and how to apply new technologies in solving business challenges rather than merely following trends. It advocates for a focused approach where companies comprehensively understand their business needs and strategically use technology to address those needs, emphasizing the necessity to avoid elevating means beyond the end goal.

The report amalgamates Dobbs's insights with industry observations, urging a recalibration in the industry's approach, prioritizing the alignment of technology and methodologies with actual business objectives for meaningful and successful outcomes.

\pagebreak
\section{Introduction}

\pagebreak
%\input{problem1.tex}
%\pagebreak
%\section{References}
    \begin{enumerate}[label={\textbf{Step}}]
      \item Item 1 \url{Item1.com}
      \item Item 2
      \item Item 3
    \end{enumerate}
%\pagebreak
%\strut \\
    \textbf{\large Contributions}
    \strut \\

    \includegraphics[width=6in,height=6.5in]{METRICSTICS/media/Contributions.png}

    \url{https://docs.google.com/document/d/1rRepmLwkqzTn-p66eA0qnV5Yr3qtT0x55YTpZvz9_jo/edit?usp=sharing}
    \strut \\
\end{document}